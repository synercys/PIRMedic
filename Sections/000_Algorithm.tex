\renewcommand\algorithmiccomment[1]{%
{\it /* {#1} */} %
}
\renewcommand{\algorithmicrequire}{\textbf{Input: }}
\renewcommand{\algorithmicensure}{\textbf{Output: }}
\renewcommand{\algorithmicforall}{\textbf{for each}}
\begin{algorithm}%[H]
\footnotesize
	\begin{algorithmic}[1]
	        \STATE \algorithmicrequire{Given the \aout signal, Supervised learning model $M$, window size $W$}
	        \STATE \algorithmicensure{Return if the sensor is working or faulty with the type of fault, if applicable}
            \IF {No occupancy}
                \STATE  \COMMENT{Use variance-based threshold to determine faults }
                \STATE $var \gets variance (A_{out}[1:W])$
                \IF{$var$ < $T_L$}
                    \RETURN \textsc{Fault: Lens cap is covered}
                \ELSIF{$var$ > $T_H$}
                    \RETURN \textsc{Fault: Lens cap has fallen off}
                \ELSE
                    \RETURN \textsc{Working: No lens-related fault}
                \ENDIF
            \ELSE
                \STATE \COMMENT{Transform signal to frequency domain}
                \STATE ($F_1, F_2, F_3, \ldots, F_{N})$) $\gets$ FFT\_Coeff ($A_{out}[1:W]$)
                \STATE \textsc{Fault\_type} $\gets$ M($F_1, F_2, F_3, \ldots, F_{N}$)
                \RETURN \textsc{Fault\_type}
            \ENDIF
    \end{algorithmic}
\caption{PIR Sensor Lens Fault Detection Algorithm}
\label{alg:fault_detection}
\end{algorithm}