\begin{abstract}
	Passive Infra-Red (PIR) sensors are ubiquitous and have applications ranging from automatic lighting and heating control in smart buildings,  towel dispensers in washrooms, security alarms (for intrusion detection) to human detection robots (for search and rescue). 
	%
	Unfortunately, PIR sensors are prone to failures during deployment due to reasons such as environmental damage, incorrect installation and component degradation among others that can lead to incorrect or faulty data. 
	%
	Currently, such failures are typically detected using either : \ca heavily engineered data-driven, statistical approaches that can have high false positive rates due to unseen data patterns or \cb expensive, unscalable methods that use additional hardware such as video cameras or a golden reference sensor.
	%
	In this work, we first create a \textit{taxonomy} for the most common PIR sensor failures and analyze these failures from the perspective of sensor physics.
	%
	We then present \sol --- a \textit{physics-driven, edge-based approach} to \textit{detect} and \textit{diagnose} the failures in a PIR sensor using an intrinsic hardware signal~\viz the analog output from the pyroelectric element in the sensor. 
	%
	Using this hardware signal in conjunction with frequency analysis and supervised machine learning methods, we obtain a high accuracy of $98-99 \%$ in failure detection and diagnosis. 
	%
	We evaluate our methods using \textit{multiple real-world deployments}, in four distinct locations, in different environment and usage conditions.    
\end{abstract}